\documentclass[10pt,aspectratio=1610]{beamer}
\usetheme[subsectionpage=progressbar,progressbar=frametitle,block=fill]{moloch}
\mode<presentation>

%%%%%%%%%%%%%%%%%%%%%%%%%%%%%%%%%%%%%%%%%%%%%%%%%%%%%%%%%%%%%%%%%%%%%%%%%%%%%%%%
%%%%%%%%%%%%%%%%%% personal modification of the beamer theme %%%%%%%%%%%%%%%%%%%
%%%%%%%%%%%%%%%%%%%%%%%%%%%%%%%%%%%%%%%%%%%%%%%%%%%%%%%%%%%%%%%%%%%%%%%%%%%%%%%%

% make footnotesize smaller
\let\oldfootnotesize\footnotesize
\renewcommand*{\footnotesize}{\oldfootnotesize\tiny}
% footnote without counter
\newcommand\blfootnote[1]{%
  \begingroup
  \renewcommand\thefootnote{}\footnote{#1}%
  \addtocounter{footnote}{-1}%
  \endgroup
}
% footnotetext without counter
\newcommand\blfootnotetext[1]{%
  \begingroup
  \renewcommand\thefootnotetext{}\footnotetext{#1}%
  \addtocounter{footnotetext}{-1}%
  \endgroup
}

%%%%%%%%%%%%%%%%%%%%%%%%%%%%%%%%%%%%%%%%%%%%%%%%%%%%%%%%%%%%%%%%%%%%%%%%%%%%%%%%
%%%%%%%%%%%%%%%%%%%%%%%%%%%%%%%%%%%% table %%%%%%%%%%%%%%%%%%%%%%%%%%%%%%%%%%%%%
%%%%%%%%%%%%%%%%%%%%%%%%%%%%%%%%%%%%%%%%%%%%%%%%%%%%%%%%%%%%%%%%%%%%%%%%%%%%%%%%
\usepackage{booktabs}

% ref: https://tex.stackexchange.com/a/614273/240783
\usepackage{tabularray}
\UseTblrLibrary{booktabs}

%%%%%%%%%%%%%%%%%%%%%%%%%%%%%%%%%%%%%%%%%%%%%%%%%%%%%%%%%%%%%%%%%%%%%%%%%%%%%%%%
%%%%%%%%%%%%%%%%%%%%%%%%%%%%%%%%%%% caption %%%%%%%%%%%%%%%%%%%%%%%%%%%%%%%%%%%%
%%%%%%%%%%%%%%%%%%%%%%%%%%%%%%%%%%%%%%%%%%%%%%%%%%%%%%%%%%%%%%%%%%%%%%%%%%%%%%%%
\usepackage{caption}
\captionsetup{font={scriptsize},
              labelfont={scriptsize},
              textfont={scriptsize},
              % hypcap=false,
              % format=hang,
              % margin=1cm
             }

%%%%%%%%%%%%%%%%%%%%%%%%%%%%%%%%%%%%%%%%%%%%%%%%%%%%%%%%%%%%%%%%%%%%%%%%%%%%%%%%
%%%%%%%%%%%%%%%%%%%%%%%%%%%%%%%%%%%% glyph %%%%%%%%%%%%%%%%%%%%%%%%%%%%%%%%%%%%%
%%%%%%%%%%%%%%%%%%%%%%%%%%%%%%%%%%%%%%%%%%%%%%%%%%%%%%%%%%%%%%%%%%%%%%%%%%%%%%%%
\usepackage{fontawesome5}
\usepackage{pifont}
\newcommand{\cmark}{\ding{51}}
\newcommand{\xmark}{\ding{55}}
\usepackage{romannum}

%%%%%%%%%%%%%%%%%%%%%%%%%%%%%%%%%%%%%%%%%%%%%%%%%%%%%%%%%%%%%%%%%%%%%%%%%%%%%%%%
%%%%%%%%%%%%%%%%%%%%%%%%%%%%%%%%%%%% color %%%%%%%%%%%%%%%%%%%%%%%%%%%%%%%%%%%%%
%%%%%%%%%%%%%%%%%%%%%%%%%%%%%%%%%%%%%%%%%%%%%%%%%%%%%%%%%%%%%%%%%%%%%%%%%%%%%%%%
\usepackage{xcolor}
    % ref: http://zhongguose.com
    \definecolor{kongquelan}{RGB}{14,176,201}
    \definecolor{koushaolv}{RGB}{93,190,138}
    \definecolor{yingwulv}{RGB}{91,174,35}
    \definecolor{shenhuilan}{RGB}{19,44,51}
    \definecolor{jianniaolan}{RGB}{20,145,168}
    \definecolor{jiguanghong}{RGB}{243,59,31}
    \definecolor{xiangyehong}{RGB}{240,124,130}
    \definecolor{red}{HTML}{E9002D}
    \definecolor{amber}{HTML}{FFAA00}
    \definecolor{green}{HTML}{00B000}

%%%%%%%%%%%%%%%%%%%%%%%%%%%%%%%%%%%%%%%%%%%%%%%%%%%%%%%%%%%%%%%%%%%%%%%%%%%%%%%%
%%%%%%%%%%%%%%%%%%%%%%%%%%%%%%%%%%% graphics %%%%%%%%%%%%%%%%%%%%%%%%%%%%%%%%%%%
%%%%%%%%%%%%%%%%%%%%%%%%%%%%%%%%%%%%%%%%%%%%%%%%%%%%%%%%%%%%%%%%%%%%%%%%%%%%%%%%
% \usepackage{graphicx} % pre-loaded by beamer
    \graphicspath{{../images}}

%%%%%%%%%%%%%%%%%%%%%%%%%%%%%%%%%%%%%%%%%%%%%%%%%%%%%%%%%%%%%%%%%%%%%%%%%%%%%%%%
%%%%%%%%%%%%%%%%%%%%%%%%%%%%%% figure annotation %%%%%%%%%%%%%%%%%%%%%%%%%%%%%%%
%%%%%%%%%%%%%%%%%%%%%%%%%%%%%%%%%%%%%%%%%%%%%%%%%%%%%%%%%%%%%%%%%%%%%%%%%%%%%%%%
% \usepackage{tikz} % pre-loaded by beamer

%\annotatedFigureBoxCustom{bottom-left}{top-right}{label}{label-position}{box-color}{label-color}{border-color}{text-color}
\newcommand*\annotatedFigureBoxCustom[8]{
    \draw[#5,thick,rounded corners] (#1) rectangle (#2); 
    \node at (#4) [fill=#6,thick,shape=circle,draw=#7,inner sep=2pt,font=\sffamily,text=#8] { \textbf{#3} };
}
%\annotatedFigureBox{bottom-left}{top-right}{label}{label-position}
\newcommand*\annotatedFigureBox[4]{
    \annotatedFigureBoxCustom{#1}{#2}{#3}{#4}{red!40}{red!20}{red!20}{black}
}
\newcommand*\annotatedFigureText[4]{\node[draw=none, anchor=south west, text=#2, inner sep=0, text width=#3\linewidth,font=\sffamily] at (#1){#4};}
\newenvironment{annotatedFigure}[1]{
    \centering
    \begin{tikzpicture}
        \node[anchor=south west,inner sep=0] (image) at (0,0) { #1 };
    \begin{scope}[x={(image.south east)},y={(image.north west)}]
}
{
    \end{scope}
    \end{tikzpicture}
}
%\figureBox{bottom-left}{top-right}{color}{thickness}
\newcommand*\figureBox[4]{\draw[#3,#4,rounded corners] (#1) rectangle (#2);}

\usetikzlibrary{calc,tikzmark}
\usepackage{tcolorbox}
\colorlet{marknode_color}{cyan!30}
\colorlet{annotate_color}{cyan!60}

%%%%%%%%%%%%%%%%%%%%%%%%%%%%%%%%%%%%%%%%%%%%%%%%%%%%%%%%%%%%%%%%%%%%%%%%%%%%%%%%
%%%%%%%%%%%%%%%%%%%%%%%%%%%%% pdf, svg, animation %%%%%%%%%%%%%%%%%%%%%%%%%%%%%%
%%%%%%%%%%%%%%%%%%%%%%%%%%%%%%%%%%%%%%%%%%%%%%%%%%%%%%%%%%%%%%%%%%%%%%%%%%%%%%%%
\usepackage{pdfpages}

%%%%%%%%%%%%%%%%%%%%%%%%%%%%%%%%%%%%%%%%%%%%%%%%%%%%%%%%%%%%%%%%%%%%%%%%%%%%%%%%
%%%%%%%%%%%%%%%%%%%%%%%%%%%%%%%%%%%% others %%%%%%%%%%%%%%%%%%%%%%%%%%%%%%%%%%%%
%%%%%%%%%%%%%%%%%%%%%%%%%%%%%%%%%%%%%%%%%%%%%%%%%%%%%%%%%%%%%%%%%%%%%%%%%%%%%%%%
\usepackage{lipsum}

%%%%%%%%%%%%%%%%%%%%%%%%%%%%%%%%%%%%%%%%%%%%%%%%%%%%%%%%%%%%%%%%%%%%%%%%%%%%%%%%
%%%%%%%%%%%%%%%%%%%%%%%%%%%%%%%%%% reference %%%%%%%%%%%%%%%%%%%%%%%%%%%%%%%%%%%
%%%%%%%%%%%%%%%%%%%%%%%%%%%%%%%%%%%%%%%%%%%%%%%%%%%%%%%%%%%%%%%%%%%%%%%%%%%%%%%%
\usepackage{url}
\usepackage[natbib=true, backend=biber, style=authoryear-icomp, useprefix=true, style=ieee]{biblatex}
\addbibresource{references.bib}
\AtBeginBibliography{\scriptsize}

\usepackage{hyperref}
\hypersetup{
    colorlinks=true,
    linkcolor=.,
    anchorcolor=.,
    filecolor=.,
    menucolor=.,
    runcolor=.,
    urlcolor=cyan,
    citecolor=.,
}

\title{Information Theory}
\subtitle{Study notes}
\author{Shuqi XIAO}

\begin{document}
\maketitle
\begin{frame}{Outline}
	\tableofcontents
\end{frame}

\section{Basic Concepts}
\begin{frame}{Self-information: Motivation}
	\vspace*{\fill}
	\par What is information?
	\vspace*{\fill}
	\par How can we define ``informational value'' of ``messages''?
	\vspace*{1.5ex}
	\begin{itemize}
		\setlength{\itemsep}{1.5ex}
		\item Messages are defined as \textbf{events} in Probability Theory.
		\item Informational value is defined by \textbf{three axioms}.
	\end{itemize}
	\blfootnote{Why is it called `self-information' instead of a simple `information'? Stay tuned.}
	\vspace*{\fill}
\end{frame}

\begin{frame}{Self-information: Definition (\Romannum{1})}
	\vspace*{\fill}
	\begin{enumerate}
		\setlength{\itemsep}{1.5ex}
		\item A certain event carries no information.
		\item The less probable an event is, the more information it yields.
		\item If two independent events occur, the total information is the sum of each.
	\end{enumerate}
	\vspace*{\fill}
	\begin{definition}[the self-information of an event]
		For an event \(X\) with the probability measure \(P\), the self-information is
		\begin{equation}
			I\left(X\,\right) := -\log \tikzmarknode{base}{\colorbox{cyan!30}{\(_{b}\)}} \left[ P\left(X\,\right) \right]
		\end{equation}
		\begin{figure}[htbp]
			\centering
			\begin{tikzpicture}[overlay,remember picture,>=stealth,nodes={align=left,inner ysep=1pt},<-]
				\path (base.south) ++ (0em,-0.5em) node[anchor=north east,color=cyan!60] (base_annotate) {\scriptsize{\(<1\), base}};
				\draw [color=cyan!60] (base.south) |- (base_annotate.south west);
			\end{tikzpicture}
		\end{figure}
	\end{definition}
	\blfootnote{If \(b=2\), the unit is the shannon (symbol \(\mathrm{Sh}\)), often called a `bit'; if \(b=\mathrm{e}\), the unit is the natural unit of information (\(\mathrm{nat}\)); \(b=10\), the hartley (\(\mathrm{Hart}\)).}
	\vspace*{\fill}
\end{frame}

\begin{frame}{Self-information: Definition (\Romannum{2})}
	\vspace*{\fill}
	\begin{proof}[Proof (Axiom 3)]
		\begin{equation}
			I\,\left(XY\,\right) = - \log \left[P(XY\,)\right] = - \log \left[P(X\,)P(Y\,)\right] = - \log \left[P(X\,)\right] - \log \left[P(Y\,)\right] = I\,\left(X\,\right) + I\,\left(Y\,\right)
		\end{equation}
	\end{proof}
	\vspace*{\fill}
	\begin{definition}[the self-information of a random variable]
		For a random variable \(X\) with the probability measure \(P\), the self-information is a function of the random variable,
		\begin{equation}
			I_X\left(x\,\right) := -\log_b\left[ P\left(X=x\,\right) \right]
		\end{equation}
	\end{definition}
	\vspace*{\fill}
\end{frame}

\begin{frame}{Entropy}
	\vspace*{\fill}
	What is the average level of information needed to describe a random variable?
	\vspace*{\fill}
	\begin{definition}[entropy]
		For a discrete random variable \(X\) with probability distribution \(p:\mathcal{X} \mapsto [0,1]\), the entropy is
		\begin{equation}
			H\,(X) := \mathbb{E} \left(I\,(x\,)\right) = -\sum_{x \,\in\mathcal{X}} p\,(x\,) \log \left[\,p\,(x\,)\right]
		\end{equation}
	\end{definition}
	\vspace*{\fill}
	\blfootnote{What is the average level of information that a random variable can offer?}
\end{frame}

\begin{frame}{Conditional Entropy}
	\vspace*{\fill}
	What is the average level of information needed to describe one random variable if another random variable is known?
	\vspace*{\fill}
	\begin{definition}[conditional entropy]
		\begin{equation}
			H\,\left(Y\,\vert X\,\right) = -\sum_{x\,\in \mathcal{X}, y\,\in \mathcal{Y}} p\left(x, y\right) p\left(y\,\vert x\right)
		\end{equation}
	\end{definition}
	\vspace*{\fill}
\end{frame}

\begin{frame}{Relative Entropy}
	\vspace*{\fill}
	On average, how much information do you need to realize that you previously misidentified one random variable \(P\) as another \(Q\)?
	\vspace*{\fill}
	\begin{definition}[relative entropy; Kullback–Leibler divergence]
		For discrete probability distributions \(P\) and \(Q\) defined on the same sample space \(\mathcal{X}\), the relative entropy from \(Q\) to \(P\) is
		\begin{equation}
			D_{KL} (P\,\Vert Q) = -\sum_{x\,\in \mathcal{X}} P(x\,) \log \left( \frac{Q(x\,)}{P(x\,)}\right)
		\end{equation}
	\end{definition}
	\vspace*{\fill}
	\blfootnote{On average, how much information do you gain if you are told that you previously misidentified one random variable \(P\) as another \(Q\)?}
\end{frame}

\begin{frame}{Mutual-information}
	\vspace*{\fill}
	How much information can we obtain about one random variable by observing the other random variable?
	\vspace*{\fill}
	\begin{definition}[mutual-information]
		TODO
	\end{definition}
\end{frame}

\section{References}
\begin{frame}[allowframebreaks]
	\frametitle{References}
	\nocite{*}
	\printbibliography[heading=none]
\end{frame}
\end{document}
